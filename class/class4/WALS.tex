\documentclass[12pt]{beamer}

\usetheme{}
\usefonttheme[stillsansserifsmall]{serif}
\usefonttheme[onlylarge]{structurebold}
%\usecolortheme[RGB={10,120,100}]{structure}
\setbeamertemplate{navigation symbols}{}

\usepackage[round]{natbib}
\usepackage[english]{babel}

\usepackage{ulem}

%\usepackage{lateL.rowval[L.colptr[i]+1xsym}
\usepackage{amsmath}
\usepackage{amssymb}
\usepackage{amsthm}
\usepackage{bbm}
\usepackage{amsfonts}
\usepackage{epsfig}
\usepackage{graphicx}
\usepackage{setspace}
\usepackage{color}
\usepackage{colordvi}
\usepackage{subcaption}
\usepackage{xspace}
\usepackage{algorithmic}

\usepackage{color}

%% Macros

\newcommand{\bmat}[1]{\begin{bmatrix}#1\end{bmatrix}}
\newcommand{\RR}{\mathbb{R}}
\newcommand{\T}{^T\!}
\newcommand{\ones}{\mathbbm{1}}
\DeclareMathOperator*{\argmin}{arg\,min}

\renewcommand*{\bibfont}{\footnotesize}
%%

\title[]{Weighted Alternating Least-Squares for Low-Rank Completion}
\author[Ron Estrin, Brad Nelson (ICME)]{CME 258}

\begin{document}

\frame{\titlepage}

\begin{frame}
Suppose that you want to recommend movies/tv shows/music based on users' viewing habits.
\\~

You have $A \in \RR^{m \times n}$ where
\begin{itemize}
\item each row $i$ corresponds to a user,
\item each column $j$ corresponds to an item (e.g. movie),
\end{itemize}
so that
$$ A_{ij} = \begin{cases} \mbox{user $i$'s rating of movie $j$} & \mbox{if $i$ watched $j$} \\ \mbox{unknown} & \mbox{otherwise.} \end{cases}$$
Want to ``complete'' the matrix by determining the unknown entries (predict a user's ratings for movies they have not yet watched).
\end{frame}

\begin{frame}
Two possible goals:
\\~
\begin{itemize}
\item Given a user's viewing history, suggest new movies to watch.
\item Given a movie, determine similar movies (nearest neighbours).
\end{itemize}
\end{frame}

\begin{frame}
Assume $A \approx UV^T$ for $U \in \RR^{m \times r}$, $V \in \RR^{n \times r}$ with $r \ll m,n$.
\\~

Arguably reasonable assumption:
\begin{itemize}
\item Can think of movies as being a ``linear combination'' of much fewer genres.
\item Can think of users as being a ``linear combination'' of people who like certain genres.
\end{itemize}
\end{frame}

\begin{frame}
Let $U = \bmat{u_1^T \\ \vdots \\ u_m^T}$, $V = \bmat{v_1^T \\ \vdots \\ v_n^T}$, and $ W_{ij} = \begin{cases} 1 & (i,j) \mbox{ observed} \\ 0 & \mbox{otherwise.} \end{cases}$
\\~
\\~

Then we solve
\begin{align*}
&\min_{U,V} \frac{1}{2} \sum_{(i,j) \mbox{ observed}} (A_{i,j} - u_i^T v_j)^2 + \lambda \left( \sum_{i=1}^m \|u_i\|_2^2 + \sum_{j=1}^n \|v_j\|_2^2 \right) \\~\\
=&\min_{U,V} \frac{1}{2}\|W \circ (A - UV^T)\|^2_F + \lambda \left( \|U\|_F^2 + \|V\|_F^2 \right),
\end{align*}
where $(A \circ B)_{ij} = A_{ij} B_{ij}$.
\end{frame}

\begin{frame}
We'll solve this problem via alternating minimization:
\begin{itemize}
\item Fix $U$ and solve for $V$, then
\item Fix $V$ and solve for $U$.
\end{itemize}
When one factor is fixed, this becomes regular linear least-squares.
\\~

Notice that this is equivalent to solving
\begin{align*}
\min_{v_1, \dots, v_n} \sum_{j=1}^n \frac{1}{2} \|W_{:,j} \circ (A_{:,j} - Uv_j)\|_2^2 + \lambda \|v_j\|_2^2,
\end{align*}
so we can solve for every row of $V$ independently.
\\~

(Need to slice columns of $A$ repeatedly...)
\end{frame}

\begin{frame}
To get $v_j$, need to solve linear system
$$ \left( U^T \mbox{diag} (W_{:,j}) U + \lambda I \right) v_j = U^T A_{:,j}. $$
\\~

How can we perform this operation quickly?
\end{frame}

\begin{frame}
Notice that 
	$$U^T \mbox{diag}(W_{:,j}) U = \sum_{i: W_{i,j} = 1} u_i u_i^T, $$
(syrk!).
\\~

Similarly,
$$ U^T A_{:,j} = \sum_{i: W_{i,j} = 1} u_i A_{i,j}. $$
\end{frame}

\begin{frame}{Homework 2}
We have provided you with a basic implementation of a weighted-alternating-least-squares solver in python. Your homework is to improve its performance!
\\~

Some ideas:
\begin{itemize}
\item Use cProfile to check which operations are taking up the most time.
\item Use the correct sparse-matrix format depending on whether $U$ or $V$ is being updated.
\item Find the best way to solve the linear system. Which factorization would you use? How would you call it?
\item Avoid ``todense()'' calls.
\end{itemize}
\end{frame}

\begin{frame}
This assignment is intentionally open-ended. Do your best to improve the performance as much as possible.
\\~

We've also provided you with a method to check the quality of your factorization: you can input a movie from the MovieLens dataset, and check which movies are considered similar according to the factorization.
\end{frame}

\end{document}